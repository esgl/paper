\documentclass[12pt,a4paper]{article}
\usepackage{amsfonts}
\usepackage{amsmath}
\usepackage{amssymb}
\usepackage{geometry}
\usepackage{indentfirst}
\usepackage{pifont}
\usepackage{ulem}
\usepackage{color}
\usepackage{algorithm} 
\usepackage{algorithmicx} 
\usepackage{algpseudocode}  
\usepackage{amsmath}  
\renewcommand{\algorithmicrequire}{\textbf{Input:}}  % Use Input in the format of Algorithm  
\renewcommand{\algorithmicensure}{\textbf{Output:}} % Use Output in the format of Algorithm  

\setlength{\parindent}{0em}
\geometry{left=2.0cm,right=2.0cm,top=2.5cm,bottom=2.5cm}

\begin{document}
\title{Notes: Asynchronous Methods for Deep Reinforcement Learning\cite{MnihBMGLHSK16}}
\author{esgl Hu}
\maketitle
\section{Review}
\paragraph{}The simple online RL algorithms with deep neural networks was fundamentally \uline{unstable}.
\paragraph{}The sequence of observed data encountered by an online RL agent is non-stationary, and online RL updates are strongly correlated.
\paragraph{}\textbf{The \textcolor{red}{experience replay} has several \textcolor{blue}{drawbacks}}: \uline{it uses more memory and computation per real interaction; and it requires off-policy learning algorithms that can update from data generated by an older policy.}
\paragraph{} The loss function of one-step Q-learning is 
\begin{equation}\label{loss_function_one_step_q_learning}
L_{i}(\theta_{i}) = \mathbb{E}(r + \gamma\cdot max_{a^{'}}Q(s^{'}, a^{'}; \theta_{i-1}) - Q(s, a; \theta_{i}))^{2}
\end{equation}
where $s^{'}$ is the state encounted after state $s$. \uline{One drawback of using one-step method is that obtaining \textcolor{red}{a reward $r$ only directly affects the value of the state action pair $s, a$ }that led to the reward. The values of other state action pairs are affected only indirectly througt the updated value $Q(s, a)$. This can make the learning process slow since many updates are required the propagate a reward to the relevant preceding states and actions.} One way of propagating rewards faster is by using $n$-step returns. In $n$-step Q-learning, $Q(s, a)$ is updated toward the $n$-step return defined as \uline{$r_{t} + \gamma r_{t+1} + \cdots + \gamma^{n-1}r_{t+n-1} + \text{max}_{a}\gamma^{n}Q(s_{t+n}, a)$}.

\paragraph{}Policy-based model-free methods directly parameterize the policy $\pi(a|s;\theta)$ and update the parameters $\theta$ by performing, typically approximate, gradient ascent on $\mathbb{E}[R_{t}]$. For example, standard \textbf{REINFORCE} updates the policy parameters $\theta$ in the direction $\nabla_{\theta}\text{log}\pi(a_{t}|s_{t}; \theta)R_{t}$, which is an unbiased estimate of $\nabla_{\theta}\mathbb{E}[R_{t}]$. It is possible to reduce the variance of this estimate while keeping it unbiased by subtracting a learned function of the state $b_{t}(s_{t})$, known as a baseline from the return. The resulting gradient is \uline{$\nabla_{\theta}log\pi(a_{t}|s_{t}; \theta)(R_{t} - b_{t}(s_{t}))$}.

\paragraph{}A learned estimate of the value function is commonly used as the baseline $b_{t}(s_{t}) \approx V^{\pi}(s_{t})$ leading to a much lower variance estimate of the policy gradient. When an approximate value function is used as the baseline, the quantity $R_{t} - b_{t}$ used to scale the policy gradient can be seen as an estimate of the \textit{advantage} of the action $a_{t}$ in state $s_t$, or $A(a_{t}, s_{t}) = Q(a_{t}, s_{t}) - V(s_{t})$


\section{Asynchronous one-step Q-learning}

\begin{algorithm} 
	\caption{Asynchronous one-step Q-learning - pseudocode for each actor-learner thread.}
	\label{asynchronous_one_step_q_learning}
	\begin{algorithmic}[1]
		\State \textit{// Assume global shared $\theta, \theta^{-}$, and counter $T = 0$.}
		\State Initialize thread step counter $t \leftarrow 0$
		\State Initialize target network weights $\theta^{-} \leftarrow \theta$
		\State Initialize network gradients $d\theta \leftarrow 0$
		\State Get initial state $s$
		\Repeat
			\State Take action $a$ with $\epsilon$-greedy policy based on $Q(s, a; \theta)$
			\State Receive new state $s^{'}$ and reward $r$
			\State $y = \left\lbrace
				\begin{array}{lcl}
					r & & \text{for terminal }s^{'} \\
					r + \gamma max_{a^{'}}Q(s^{'}, a^{'}; \theta^{-}) & & \text{for non-terminal }s^{'}
				\end{array}	\right.$						
			
			\State Accumulate gradients wrt : $d\theta \leftarrow d\theta + \frac{\partial(y - Q(s, a; \theta))^{2}}{\partial\theta}$
			\State $s = s^{'}$
			\State $T \leftarrow T + 1$ and $t \leftarrow t + 1$
			\If {$T$ mod $I_{target} == 0$}
				\State Update the target network $\theta^{-} \leftarrow \theta$
			\EndIf
			\If{ $t$ mod $I_{asyncUpdate} == 0$ or $s$ is terminal}
				\State Perform asynchronous update of $\theta$ using $d\theta$.
				\State Clear gradients $d\theta \leftarrow 0$.
			\EndIf
		\Until{$T > T_{max}$}
	\end{algorithmic}
\end{algorithm}

\section{Asynchronous one-step Sarsa}
\paragraph{}The asynchronous one-step Sarsa algorithm is the same as asynchronous one-step Q-learning as given in algorithm {\ref{asynchronous_one_step_q_learning}} except that is uses a different target value for $Q(s, a)$. The target value used by one-step Sarsa is $r + \gamma Q(s^{'}, a^{'}; \theta^{-})$ where $a^{'}$ is the action taken in state $s^{'}$.



\section{Asynchronous n-step Q-learning}

\begin{algorithm} 
	\caption{Asynchronous one-step Q-learning - pseudocode for each actor-learner thread.}
	\label{asynchronous_n_step_q_learning}
	\begin{algorithmic}[1]
		\State \textit{// Assume global shared parameter vector $\theta$}
		\State \textit{// Assume global shared target parameter vector $\theta^{-}$}
		\State \textit{// Assume global shared counter $T = 0$}
		\State Initialize thread step count $t \leftarrow 1$
		\State Initialize target network parameters $\theta \leftarrow \theta$
		\State Initialize thread-specific parameters $\theta^{-} \leftarrow \theta$
		\State Initialize network gradients $d\theta \leftarrow 0$
		\Repeat 
			\State Clear gradient $d\theta \leftarrow 0$
			\State Synchronous thread-specific parameters $\theta^{'} = \theta$
			\State $t_{target} = t$
			\State Get state $s_{t}$
			\Repeat
				\State Take action $a_{t}$ accordin to the $\epsilon$-greedy policy based on $Q(s_{t}, a_{t};\theta^{'})$
				\State Receive reward $r_{t}$ and new state $s_{t+1}$
				\State $t \leftarrow t + 1$
				\State $T \leftarrow T + 1$
			\Until {terminal $s_{t}$ or $t - t_{target} == t_{max}$}
			\State $R = \left\lbrace
				\begin{array}{lcl}
					0 & & \text{for terminal }s_{t} \\
					max_{a}Q(s^{t}, a; \theta^{-}) & & \text{for non-terminal }s_{t}
				\end{array}	\right.$	
			\For {$i \in {t - 1, ..., t_{start}}$}
				\State $R \leftarrow r_{i} + \gamma R$
				\State Accumulate gradients wrt $\theta^{'} : d\theta + \frac{\partial(R - Q(s_{i},a_{i}; \theta^{'}))^{2}}{\partial{\theta^{'}}}$
			\EndFor	
			\State Performance asynchronous update of $\theta$ using $d\theta$.
			\If $T mod I_{target} == 0$
				\State $\theta^{-} \leftarrow \theta$
			\EndIf
		\Until {$T > T_{max}$}		
		
	\end{algorithmic}
\end{algorithm}

\section{Asynchronous advantage actor-critic}
\begin{algorithm} 
	\caption{Asynchronous advantage actor-critic - pseudocode for each actor-learner thread.}
	\label{asynchronous_advantage_actor_critic}
	\begin{algorithmic}[1]
		\State \textit{// Assume global shared parameter vector $\theta$ and $\theta_{v}$ and global shared counter $T = 0$}
		
		\State \textit{// Assume thread-specific parameters $\theta^{'}$ and $\theta_{v}^{'}$}
		\State Initialize thread step counter $t \leftarrow 1$
		\Repeat 
			\State Reset gradients: $d\theta \leftarrow 0$ and $d\theta_{v} \leftarrow 0$.
			\State Synchronous thread-specific parameters $\theta^{'} = \theta$ and $\theta_{v}^{'} = \theta_{v}$
			\State $t_{target} = t$
			\State Get state $s_{t}$
			\Repeat
				\State Perform $a_t$ according to policy $\pi(a_{t}|s_{t};\theta^{'})$
				\State Receive reward $r_{t}$ and new state $s_{t+1}$
				\State $t \leftarrow t + 1$
				\State $T \leftarrow T + 1$
			\Until {terminal $s_{t}$ or $t - t_{target} == t_{max}$}
			\State $R = \left\lbrace
				\begin{array}{lcl}
					0 & & \text{for terminal }s_{t} \\
					V(s_{t}, \theta_{v}^{'}) & & \text{for non-terminal }s_{t} \text{//Bootstrap from last state}
				\end{array}	\right.$	
			\For {$i \in {t - 1, ..., t_{start}}$}
				\State $R \leftarrow r_{i} + \gamma R$
				\State Accumulate gradients wrt $\theta^{'}: d\theta + \Delta_{\theta^{'}}log\pi(a_{i}|s_{i};\theta^{'})(R - V(s_{i};\theta_{v}^{'}))$
				\State Accumulate gradients wrt $\theta_{v}^{'}: d\theta_{v} + \frac{\partial{(R-V(s_{i}; \theta_{v}^{'}))^{2}}}{\partial\theta_{v}^{'}}$
				
			\EndFor	
			\State Perform asynchronous update of $\theta$ using $d\theta$ and of $\theta_{v}$ using $d\theta_{v}$
		\Until {$T > T_{max}$}		
		
	\end{algorithmic}
\end{algorithm}

\bibliographystyle{plain}
\bibliography{reference}
\end{document}